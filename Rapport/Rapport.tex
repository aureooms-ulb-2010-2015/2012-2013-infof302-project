%----------------------------------------------------------------------------------------
%	DOCUMENT CONFIGURATIONS
%----------------------------------------------------------------------------------------

\documentclass[a4paper,12pt]{report}
\usepackage[francais]{babel}
\usepackage[utf8]{inputenc}
\usepackage[T1]{fontenc}
\usepackage{ucs}
\usepackage{url}
\usepackage{graphicx}
\usepackage[table]{xcolor}
\usepackage{mathtools,amssymb,amsthm}%
\usepackage[left=2.5cm,top=2cm,right=2.5cm,nohead,nofoot]{geometry}
\usepackage{pdfpages}
\usepackage[table]{xcolor}
\usepackage{algorithm}
\usepackage{algpseudocode}
\linespread{1.1}
%%%%%%%%%%%%%%%%%
\makeatletter
\newif\if@borderstar
\def\bordermatrix{\@ifnextchar*{%
\@borderstartrue\@bordermatrix@i}{\@borderstarfalse\@bordermatrix@i*}%
}
\def\@bordermatrix@i*{\@ifnextchar[{\@bordermatrix@ii}{\@bordermatrix@ii[()]}}
\def\@bordermatrix@ii[#1]#2{%
\begingroup
\m@th\@tempdima8.75\p@\setbox\z@\vbox{%
\def\cr{\crcr\noalign{\kern 2\p@\global\let\cr\endline }}%
\ialign {$##$\hfil\kern 2\p@\kern\@tempdima & \thinspace %
\hfil $##$\hfil && \quad\hfil $##$\hfil\crcr\omit\strut %
\hfil\crcr\noalign{\kern -\baselineskip}#2\crcr\omit %
\strut\cr}}%
\setbox\tw@\vbox{\unvcopy\z@\global\setbox\@ne\lastbox}%
\setbox\tw@\hbox{\unhbox\@ne\unskip\global\setbox\@ne\lastbox}%
\setbox\tw@\hbox{%
$\kern\wd\@ne\kern -\@tempdima\left\@firstoftwo#1%
\if@borderstar\kern2pt\else\kern -\wd\@ne\fi%
\global\setbox\@ne\vbox{\box\@ne\if@borderstar\else\kern 2\p@\fi}%
\vcenter{\if@borderstar\else\kern -\ht\@ne\fi%
\unvbox\z@\kern-\if@borderstar2\fi\baselineskip}%
\if@borderstar\kern-2\@tempdima\kern2\p@\else\,\fi\right\@secondoftwo#1 $%
}\null \;\vbox{\kern\ht\@ne\box\tw@}%
\endgroup
}
\makeatother
%%%%%%%%%%%%%%%%%
\newcommand\black{\cellcolor{black}}
\newcommand\grey{\cellcolor{black!50}}

%%%%%%%%%%%%%%%%%
\begin{document}


\setlength\parindent{0pt} % Removes all indentation from paragraphs


\begin{titlepage}
\begin{center}
\textbf{\textsc{UNIVERSIT\'E LIBRE DE BRUXELLES}}\\
\textbf{\textsc{Faculté des Sciences}}\\
\textbf{\textsc{Département d'Informatique}}
\vfill{}\vfill{}
\begin{center}{\Huge INFO-F-302 - Logique Informatique \\Projet: Le jeu Pattern et Utilisation de MiniSAT}\end{center}{\Huge \par}
\begin{center}{\large \textsc{Ooms} Aurélien, \textsc{Sonnet} Jean-Baptiste}\end{center}{\Huge \par}
\vfill{}\vfill{}
\vfill{}\vfill{}\enlargethispage{3cm}
\textbf{Année académique 2012~-~2013}
\end{center}
\end{titlepage}




\tableofcontents
\newpage


\chapter{Énumération 3,2}

\section{Problème et notation}
\subsection{Grille}
Le problème est présenté sous forme d'une grille $3\times3$ contenant au $max$ 2 contraintes par ligne ou colonne.\\

Soit une matrice $3\times3$,  
$$\begin{bmatrix} x_{0,0} & x_{0,1} & x_{0,2} \\ x_{1,0} & x_{1,1} & x_{1,2} \\ x_{2,0} & x_{2,1} & x_{2,2}\end{bmatrix}$$ 
où chacune des cases $x_{i,j}$ prendra potentiellement une des 3 valeurs: 
$\{0,1,-1\}$, respectivement l'inconnu, le noir, le blanc.\\

On aura par exemple comme problème à résoudre:
\begin{center}
$\bordermatrix[{[]}]{
	\text{ }	 
		& 2		& 1		& 2		\cr
1	 	& 1		& -1 	& 0		\cr
1\; 1  	& 0 	& 0 	& 0		\cr
2    	& 0 	& 0 	& 0		\cr
} $
\begin{tabular}{|c|c|c|}
\hline 
\black• &   & \grey?  \\ 
\hline 
\grey? & \grey?  & \grey? \\ 
\hline 
\grey? & \grey? & \grey? \\ 
\hline 
\end{tabular}
\end{center}

Selon les contraintes précisées, la solution devra donner pour toutes les cases inconnues une valeur de $1$ ou $-1$:\\
\begin{center}
$\bordermatrix[{[]}]{
	\text{ }	 
		& 2		& 1		& 2		\cr
1	 	& 1		& -1 	& -1	\cr
1\; 1  	& 1 	& -1 	& 1		\cr
2    	& -1 	& 1 	& 1		\cr
}$
\begin{tabular}{|c|c|c|}
\hline 
\black• &  &  \\ 
\hline 
\black• &  & \black• \\ 
\hline 
 & \black• & \black• \\ 
\hline 
\end{tabular}
\end{center}
\subsection{Cases}
Chaque case est représentée par une variable $x$, de sorte que la case à la $i^{eme}$ ligne et à la $j^{eme}$ colonne se note $x_{i,j}$

\section{Parcours}
\begin{algorithm}
\caption{Énumération selon les contraintes de lignes et de colonnes}
\begin{algorithmic}

\For{ligne $i \to n$}
		\For{colonne $j \to n$}
			\If{$x_{i,j}==-1$}
				\State Créer une nouvelle $clause$ avec $-x_{i,j}$
			\EndIf
			\If{$x_{i,j}==1$}
				\State Créer une nouvelle $clause$ avec $x_{i,j}$
			\EndIf
			\If{$x_{i,j}==0$}
				\If{$\not\exists \; clause_i$}
				 \State Créer $clause_i$
				\EndIf
				\State	Ajouter ($\vee$) $x_{i,j}$ à la $clause_i$ 
			\EndIf
		\State Joindre ($\wedge$) les $clauses$
		\EndFor
		\If{$\exists \; clause_i$}
			\State Joindre ($\wedge$) la $clause_i$
		\EndIf
\EndFor
\For{colonne $j \to n$}		
		\For{ligne $i \to n$}
			\If{$x_{i,j}==-1$}
				\State Créer une nouvelle $clause$ avec $-x_{i,j}$
			\EndIf
			\If{$x_{i,j}==1$}
				\State Créer une nouvelle $clause$ avec $x_{i,j}$
			\EndIf
			\If{$x_{i,j}==0$}
				\If{$\not\exists \; clause_j$}
				 \State Créer $clause_j$
				\EndIf
				\State	Ajouter ($\vee$) $x_{i,j}$ à la $clause_j$ 
			\EndIf
		\State Joindre ($\wedge$) les $clauses$
		\EndFor
		\If{$\exists \; clause_j$}
			\State Joindre ($\wedge$) la $clause_j$
		\EndIf
\EndFor
\end{algorithmic}
\end{algorithm}
Représentation du problème en fonction normale conjonctive par énumération.
 


	
						
\end{document}