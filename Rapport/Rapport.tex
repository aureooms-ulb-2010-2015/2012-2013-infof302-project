%----------------------------------------------------------------------------------------
%	DOCUMENT CONFIGURATIONS
%----------------------------------------------------------------------------------------

\documentclass[a4paper,12pt]{report}
\usepackage[francais]{babel}
\usepackage[utf8]{inputenc}
\usepackage[T1]{fontenc}
\usepackage{ucs}
\usepackage{url}
\usepackage{graphicx}
\usepackage[table]{xcolor}
\usepackage{mathtools,amssymb,amsthm}%
\usepackage[left=2.5cm,top=2cm,right=2.5cm,nohead,nofoot]{geometry}
\usepackage{pdfpages}
\usepackage[table]{xcolor}
\usepackage{color}
\usepackage{algorithm}
\usepackage{algpseudocode}
\usepackage{array}
\linespread{1.1}
%%%%%%%%%%%%%%%%%
\makeatletter
\newif\if@borderstar
\def\bordermatrix{\@ifnextchar*{%
\@borderstartrue\@bordermatrix@i}{\@borderstarfalse\@bordermatrix@i*}%
}
\def\@bordermatrix@i*{\@ifnextchar[{\@bordermatrix@ii}{\@bordermatrix@ii[()]}}
\def\@bordermatrix@ii[#1]#2{%
\begingroup
\m@th\@tempdima8.75\p@\setbox\z@\vbox{%
\def\cr{\crcr\noalign{\kern 2\p@\global\let\cr\endline }}%
\ialign {$##$\hfil\kern 2\p@\kern\@tempdima & \thinspace %
\hfil $##$\hfil && \quad\hfil $##$\hfil\crcr\omit\strut %
\hfil\crcr\noalign{\kern -\baselineskip}#2\crcr\omit %
\strut\cr}}%
\setbox\tw@\vbox{\unvcopy\z@\global\setbox\@ne\lastbox}%
\setbox\tw@\hbox{\unhbox\@ne\unskip\global\setbox\@ne\lastbox}%
\setbox\tw@\hbox{%
$\kern\wd\@ne\kern -\@tempdima\left\@firstoftwo#1%
\if@borderstar\kern2pt\else\kern -\wd\@ne\fi%
\global\setbox\@ne\vbox{\box\@ne\if@borderstar\else\kern 2\p@\fi}%
\vcenter{\if@borderstar\else\kern -\ht\@ne\fi%
\unvbox\z@\kern-\if@borderstar2\fi\baselineskip}%
\if@borderstar\kern-2\@tempdima\kern2\p@\else\,\fi\right\@secondoftwo#1 $%
}\null \;\vbox{\kern\ht\@ne\box\tw@}%
\endgroup
}
\makeatother
%%%%%%%%%%%%%%%%%
\newcommand\black{\cellcolor{black}}
\newcommand\grey{\cellcolor{black!50}}

%%%%%%%%%%%%%%%%%
\begin{document}


\setlength\parindent{0pt} % Removes all indentation from paragraphs


\begin{titlepage}
\begin{center}
\textbf{\textsc{UNIVERSIT\'E LIBRE DE BRUXELLES}}\\
\textbf{\textsc{Faculté des Sciences}}\\
\textbf{\textsc{Département d'Informatique}}
\vfill{}\vfill{}
\begin{center}{\Huge INFO-F-302 - Logique Informatique \\Projet: Le jeu Pattern et Utilisation de MiniSAT}\end{center}{\Huge \par}
\begin{center}{\large \textsc{Ooms} Aurélien, \textsc{Sonnet} Jean-Baptiste}\end{center}{\Huge \par}
\vfill{}\vfill{}
\vfill{}\vfill{}\enlargethispage{3cm}
\textbf{Année académique 2012~-~2013}
\end{center}
\end{titlepage}




\tableofcontents
\newpage


\chapter{Énumération 3,2}

\section{Problème et notation}
\subsection{Grille}
Le problème est présenté sous forme d'une grille $3\times3$ contenant au $max$ 2 contraintes par ligne ou colonne.\\

Soit une matrice $3\times3$,  
$$\begin{bmatrix} x_{0,0} & x_{0,1} & x_{0,2} \\ x_{1,0} & x_{1,1} & x_{1,2} \\ x_{2,0} & x_{2,1} & x_{2,2}\end{bmatrix}$$ 
où chacune des cases $x_{i,j}$ prendra potentiellement une des 3 valeurs: 
$\{0,1,-1\}$, respectivement l'inconnu, le noir, le blanc.\\

On aura par exemple comme problème à résoudre:
\begin{center}
$\bordermatrix[{[]}]{
	\text{ }	 
		& 2		& 1		& 2		\cr
1	 	& 1		& -1 	& 0		\cr
1\; 1  	& 0 	& 0 	& 0		\cr
2    	& 0 	& 0 	& 0		\cr
} $
\begin{tabular}{|c|c|c|}
\hline 
\black• &   & \grey?  \\ 
\hline 
\grey? & \grey?  & \grey? \\ 
\hline 
\grey? & \grey? & \grey? \\ 
\hline 
\end{tabular}
\end{center}

Selon les contraintes précisées, la solution devra donner pour toutes les cases inconnues une valeur de $1$ ou $-1$:\\
\begin{center}
$\bordermatrix[{[]}]{
	\text{ }	 
		& 2		& 1		& 2		\cr
1	 	& 1		& -1 	& -1	\cr
1\; 1  	& 1 	& -1 	& 1		\cr
2    	& -1 	& 1 	& 1		\cr
}$
\begin{tabular}{|c|c|c|}
\hline 
\black• &  &  \\ 
\hline 
\black• &  & \black• \\ 
\hline 
 & \black• & \black• \\ 
\hline 
\end{tabular}
\end{center}
\subsection{Cases}
Chaque case est représentée par une variable $x$, de sorte que la case à la $i^{eme}$ ligne et à la $j^{eme}$ colonne se note $x_{i,j}$

\section{Parcours et représentation par énumération}


On va parcourir la grille et énumérer les combinaisons possibles premièrement suivant les contraintes de lignes ensuite selon les contraintes de colonnes.\\ 

Par ligne (respectivement colonne), on crée si nécessaire une clause représentant les cas possibles quant aux cases dont on ne connait pas encore la nature.\\

Pour chaque case contenue dans une ligne donnée:
\begin{enumerate}
\item Si elle contient une information, on en créer une clause à part entière qui sera jointe aux autres.
\item Si elle ne contient pas d'informations et conditionnellement à la contrainte, on énumère les possibilités sous forme disjonctive (clause).
\end{enumerate}
On effectue de même par colonne et relativement aux contraintes de la colonne.\\

L'énumération des cas possibles se formule naturellement à l'aide de la disjonction exclusive (XOR ou $\oplus$), à chaque étape, il convient de la traduire en forme normale conjonctive (FNC).
$$ a\oplus b \equiv (a \wedge \lnot b)\vee(\lnot a \wedge b) \equiv (a\vee b)\wedge\lnot(a\wedge b)$$
\begin{displaymath}
\begin{array}{c|c|c|c|c}
   a
 & b
 & a \oplus b
 & (a\land{}\lnot{}b)\lor{}(\lnot{}a\land{}b)
  & (a\lor{}b)\land{}\lnot{}(a\land{}b) \\
\hline
0 & 0 & 0 & 0 & 0 \\
0 & 1 & 1 & 1 & 1 \\
1 & 0 & 1 & 1 & 1 \\
1 & 1 & 0 & 0 & 0 \\

\end{array}
\end{displaymath}


\begin{algorithm}
\caption{Énumération selon les contraintes de lignes et de colonnes}
\begin{algorithmic}
\For{ligne $i \to n$}\Comment{contraintes de lignes}
		\For{colonne $j \to n$}
			\If{$x_{i,j} == \bot$}
				\State Créer une nouvelle $clause$ avec $-x_{i,j}$
			\ElsIf{$x_{i,j} == \top$}
				\State Créer une nouvelle $clause$ avec $x_{i,j}$
			\ElsIf {$x_{i,j}==0$}
				\If{$\not\exists \; clause_i$}
				 \State Créer $clause_i$
				\EndIf
				\State	Ajouter ($\vee$) $x_{i,j}$ à la $clause_i$ 
			\EndIf
		\State Joindre ($\wedge$) les $clauses$
		\EndFor
		\If{$\exists \; clause_i$}
			\State Joindre ($\wedge$) la $clause_i$
		\EndIf
\EndFor

\For{colonne $j \to n$}	\Comment{contraintes de colonnes}
		\For{ligne $i \to n$}
			\If{$x_{i,j}==\bot$}
				\State Créer une nouvelle $clause$ avec $-x_{i,j}$
			\ElsIf{$x_{i,j}==\top$}
				\State Créer une nouvelle $clause$ avec $x_{i,j}$
			\ElsIf{$x_{i,j}==0$}
				\If{$\not\exists \; clause_j$}
				 \State Créer $clause_j$
				\EndIf
				\State	Ajouter ($\vee$) $x_{i,j}$ à la $clause_j$ 
			\EndIf
		\State Joindre ($\wedge$) les $clauses$
		\EndFor
		\If{$\exists \; clause_j$}
			\State Joindre ($\wedge$) la $clause_j$
		\EndIf
\EndFor
\end{algorithmic}
\end{algorithm}


\chapter{Les Bandes}
\section{Problème et notation}

Une suite de cases noires représente une \textit{bande}. 

Une bande appartient à une ligne ou colonne et commence à une position donnée. Comme il peut y avoir plusieurs bandes sur une même ligne ou colonne, chaque bande doit être identifiable.\\

On représente une bande par $$LBande_{pos,ID,start}$$
où,
\begin{description}
\item[L|C] précise s'il s'agit d'une ligne ou d'une colonne.
\item[pos] désigne le numéro de la ligne ou la colonne où est située la bande.
\item[ID] désigne, s'il existe plusieurs bandes sur la ligne ou colonne, la bande que l'on considère.
\item[start] désigne le numéro de colonne ou ligne à partir de laquelle commence la bande.\\
\end{description}

On défini deux fonctions donnant la longueur d'une bande:
\begin{description}
\item[$\mathtt{L}(pos,ID)$] donnant la longueur de la $ID^{eme}$ bande à la ligne $pos$.
\item[$\mathtt{C}(pos,ID)$] donnant la longueur de la $ID^{eme}$ bande à la colonne $pos$.\\
\end{description} 

\begin{table}[H]
   \centering
	\begin{tabular}{|c|c|c|}
	\hline 
	\black• &  &  \\ 
	\hline 
	\black• &  & \black• \\ 
	\hline 
	 & \black• & \black• \\ 
	\hline 
	\end{tabular}
\caption{\label{Grille3X3} Grille $3\times3$}
\end{table}
On aura pour la grille \ref{Grille3X3}:
$$\mathtt{LBande}_{3,1,2} = \top$$
$$L(3,2) = 2$$
et\\
$$x_{3,j} = \top, \forall j \in [3,2+\mathtt{L}(3,2)]$$
$$\wedge$$
$$x_{3,2+\mathtt{L}(3,2)+1} = \bot  $$
\section{Représentation par bandes}
\subsection{Observation: position minimale, maximale et sous-espace}

Soit une colonne d'une longueur de 8 cases avec pour contrainte 3 bandes de taille 1, 2 et 1.

La première bande pourra prendre la position minimum $x_{1,0}$ et maximum $x_{1,\alpha}$.

La valeur de $\alpha$ peut être obtenue à partir du nombre de contraintes, de la taille des différentes bandes et de la règle selon laquelle deux bandes sont séparés d'au moins une case.\\

Dans l'exemple, on obtient:
\begin{itemize}
\item[] $c = 3$, où $c$ est le nombre de contraintes
\item[] $l = 8$ est la longueur de la colonne
\item[] $L_{total} = 4$, est la longueur cumulée des bandes dans la colonne
\item[] $\alpha = l-(L_{total}+(c-1))$
\end{itemize}

\begin{table}[H]
\centering
\begin{tabular}{|l||c|c|c|}
\hline 
0&\black• &  &  \\ 
\hline 
1 & &  &  \\ 
\hline 
2 & \grey &  & \black• \\ 
\hline 
3&\grey &  & \\ 
\hline 
4&\grey &  & \grey \\ 
\hline 
5&\grey &  & \grey \\ 
\hline 
6&\grey &  & \grey \\ 
\hline 
7&\grey &  & \grey \\ 
\hline &&\\
[-1.3em]\hline
&$min$ &  & $max$ \\ 
\hline 
\end{tabular} 
\caption{\label{bande8X1} position $max$ et $min$ sur une colonne $8\times1$ avec la contrainte 1 2 1}
\end{table}


On voit dans le tableau \ref{bande8X1} qu'il est possible de connaitre \textit{a priori} le nombre restreint de positions possibles pour la première bande. 

On voit aussi que chaque possibilité délimite un \textit{sous-espace} (zone grisée) au sein duquel devront être placées les bandes restantes.

Il est alors possible d'appliquer la même analyse quant à la détermination de la position de la bande suivante au sein du sous-espace.

Une fois les positions de la seconde bande déterminées, elles délimiteront l'espace restant pour la dernière bande. 

Suite à cela, il est possible d'élaborer une méthode récursive, ou faussement récursive, au nombre d'itérations réduit, permettant d'énumérer les possibilités de placement des bandes.

\subsection{Énumération récursive}
\begin{table}[H]
\centering
\begin{tabular}{|c|c|c|l|c|l|c|c|c|}
\hline 
\grey &&\black & \multicolumn{5}{l|}{$\updownarrow \mathtt{L}(1,1) = 1$}& $x_{1,1}=\top$ \\ 
\hline 
\grey && & \multicolumn{5}{l|}{+1} & $x_{1,2}=\bot$ \\ 
\hline
\grey &&\grey &  &  & \multicolumn{3}{l|}{$\#$ itération}& $x_{1,3}=\bot$ \\ 
\hline 
\grey &&\grey &  & \black - & \multicolumn{3}{l|}{$\updownarrow \mathtt{L}(1,2) = 2$}& $x_{1,4}=\top$ \\ 
\cline{1-5}\cline{9-9}
\grey &&\grey &  & \black - & \multicolumn{3}{l|}{}& $x_{1,5}=\top$\\ 
\hline 
\grey &&\grey &  &  & \multicolumn{3}{l|}{+1} & $x_{1,6}=\bot$\\ 
\hline 
\grey &&\grey &  & \grey &  & \black - & $\updownarrow \mathtt{L}(1,3) = 1$ & $x_{1,7}=\top$ \\ 
\hline 
\grey &&\grey &  & \grey &  &  &  & $x_{1,8}=\bot$\\ 
\hline
\hline 
 $\bullet$&&$r_1$ &  & $r_1$ &  &$r_3$&  \multicolumn{2}{c|}{récursion}\\
\hline
\end{tabular}
\caption{\label{EnumBande} Énumération récursive sur une colonne $8\times1$ avec la contrainte 1 2 1}
\end{table}

 
\begin{algorithm}
\caption{Énumération des bandes}
\begin{algorithmic}
\For{ligne $i \to n$}
		\For{colonne $j \to n$}
		nop();		
		\EndFor			
\EndFor
\end{algorithmic}
\end{algorithm}

	
						
\end{document}